\documentclass[conference]{IEEEtran}
\usepackage{cite}
\usepackage{color}
\usepackage{hyperref}
\usepackage{graphicx}
\usepackage[utf8]{inputenc}
\usepackage[greek, brazil, english]{babel}

\hyphenation {}

\title{\LARGE RSA}

\author {
  Lucas Tonussi\\
  \textsc{lptonussi@gmail.com}\\
  \small{Ciência da Computação}
}

\begin{document}

\maketitle

\begin{abstract}
Respostas do questionário sobre RSA e breve documentação do código em python que
implementa o algoritmo de RSA e geração de chaves. Em sistemas Unix utiliza
entropia interna do sistema operacional, em Windows não.
\end{abstract}

\IEEEoverridecommandlockouts

\begin{keywords}
RSA, Python, Prime Numbers, Solovay-Strassen, Euler, Adrien-Legendre, PKCS\#1,
Python 3.
\end{keywords}

\IEEEpeerreviewmaketitle

\section{Introdução}

O código que acompanha esse artigo é a implementação em Python 3 do algoritmo
RSA-(de)encriptação e geração de chaves como descrito na página 267 capítulo 9
\cite{stallings} tanto de texto quanto de valores numéricos. para rodar o código
basta executar no seu terminal:

\hfil

\begin{quotation}
  \small \textit{python3} \textit{\_\_init\_\_.py} \textit{tamanho\_chave}
  \textit{acuracidade} \\

  exemplo: \textit{python3} \textit{\_\_init\_\_.py} 128 25
\end{quotation}

\hfil

Atenção o código está preparado para ser executado tanto em ambiente Windows
quanto Unix. Porém só o código só executa em Python 3. Portanto você tem que ter
o Python versão 3 instalado no seu sistema. Em sistemas Unix, o comando dd irá
ser utilizado para extrair valores aleatórios do sistema unix/linux, mais
precisamente do arquivo /dev/urandom (unsigned random numbers generator). O
programa utiliza Mersenne Twister para gerar dois primos aleatórios de 100
dígitos decimais cada \footnote{em sistemas unix, utilizará /dev/urandom como
  fonte entrópica para semente do gerador mersenne twister}. Após executar o
código, ele vai esperar que você digite um valor de entrada, podendo ser um
valor numérico ou uma \textit{string} qualquer. então aperte \textbf{enter} e
ele irá computar os resultados. Ver o arquivo DATA que são uma execução com a
entradas diferentes para serem encriptadas. No código para criptografar
múltiplos bytes eu utilizo o abordagem parecida com o que está explicado na
literatura, Página 268 - Capítulo 9 \cite{stallings}.

\section{Quantidade de bits das chaves}

O software permite que a entrada para quantidade de bits seja de 64 até 2048
bits como especificado, porém nos casos especiais de 2048 bits de tamanho de
chave o problema principal se encontra no algoritmo recursivo de Adrien-Legendre
dentro do procedimento de Solovay-Strassen que é para testar pseudo aleatórios,
a fim de encontrar números primos. O Miller Rabin também não desempenha bem
ocorrendo falha de \textit{maximum recursion depth exceeded in comparison} no
algoritmo de \textbf{gcd} dentro do Miller-Rabin. A notação de Adrien-Legendre é
um \textbf{gcd} modificado, isto é: $n \gets (\frac{random_i}{prime})$. Consta
no algoritmo em sí de Adrien-Legendre. Para o caso de 1024 bits ocorre também a
mesma falha devido a tentativa de derivação de um número muito grande, do mesmo
jeito que ocorreu com 2048 bits.

\section{Questionário}

\begin{enumerate}
  \item O que é a função Totiente de Euler $\Phi(n)$ ? Por que os expoentes são
  determinados módulo $\Phi(n)$, e não módulo n ?

    \begin{enumerate}
      \item Segundo o algoritmo RSA \cite{rsakey} o totiente vem a partir do
      resultado da multiplicação entre os dois primos grandes pseudo-aleatórios;
      providos externamente para o algoritmo RSA de geração de chaves.
      \item Por isso, a função totient de Euler tem um papel especial no 
      algoritmo RSA.
      \item Como segue da teoria que $\Phi(n) \textit{ tal que } n \textit{ é
      primo}$. é a ordem do grupo multiplicativo discreto modulo $n$, $n$ é um
      primo.
      \item $a^{\phi(n)} \equiv 1\ (\textrm{mod}\ n)$ quando n é um primo, isso
      nos leva a um teorema chamado Pequeno Teorema de Fermat.
      \item Segue que a inversa $a \leftarrow a^e\ (\textrm{mod}\ n)$ onde 
      \textbf{e} é o expoente público (ciframento), é $a \leftarrow a^d\ 
      (\textrm{mod}\ n)$ onde \textbf{d} é o expoente privado (deciframento)
      é também $d \times e \equiv 1\ (\textrm{mod}\ \phi(n))$
    \end{enumerate}

  \item O que é PKCS \#1?

    \begin{enumerate}
      \item É uma padronização, usada no algoritmo RSA (na geração da inversa
      multiplicativa) que consiste em escolher um valor $d$ que combinado na
      equação $d*e \equiv 1\ (\textrm{mod}\ \lambda)$ \cite{pkcs1}, com $\lambda
      = lcm(p - 1, q - 1)$ onde $lcm$ é o máximo multiplicador comum. Usando
      $\lambda$ ao invés de $\Phi(n)$ permite \textbf{mais escolhas} de valores
      para $d$. $\lambda$ pode também ser definido usando a função de Carmichael
      \cite{carm}, $\lambda(n)$.
      \item O método de encriptação do RSA utiliza o PKCS \#1 \textit{v1} (é um
      esquema de enxerto (padding) próprio do algoritmo que aumenta a
      aleatoriedade e adiciona estrutura a informação (mensagem criptografada
      por \textbf{RSA}) seguindo esse protocolo é possível dizer diretamente se
      uma mensagem é válida ou não para o algoritmo.
      \item Pois os blocos da mensagem não podem ser maiores que $n = pq$, ao
      dividir em blocos, existe uma forma especial de \textit{padding} para
      complicar mais ainda a criptografia.
      \item O meu programa em Python. Não implementa o PKCS \#1 em nenhum
      momento.
    \end{enumerate}

  \item Como o RSA pode ser usado para ciframento de dados? E como seria usado
  para assinatura?

    \begin{enumerate}
      \item A chave pública é usada para encriptar a mensagem. O expoente
      público que é gerado no algoritmo (na parte de geração de chaves) pode ser
      livremente distribuído para que alguém transforme uma assinatura,
      documento, mensagem, valores, em um valor confuso e fortemente
      não-derivável. E quem receber a mensagem pode decifrar usando a chave
      privada o expoente privado que é resultado da inversa multiplicativa entre
      o expoente público e o totiente de Euler de $n = p \times q$.
    \end{enumerate}

  \item Quais ataques são conhecidos do RSA ? Listar os conhecidos,
  referenciando-os na literatura, e explique e mostre exemplos de pelo menos 2
  dos ataques;

    \begin{enumerate}
      \item A segurança do RSA tem 5 principais abrangências de ataque ao RSA
        \begin{enumerate}
          \item \textbf{Brute force}: Envolve em tentar todas as possíveis
          \textit{PRIVATE} \textit{KEYS}. Página 275 \cite{stallings}
          \item \textbf{Mathematical attacks}: Existem muitas variações, mas
          consiste em fatorar a multiplicação entre dois primos $p \times q$.
          Página 275 \cite{stallings}
          \item \textbf{Timing attacks}: Esse ataque depende do tempo em que o
          algoritmo RSA demora para execução. Kocher, página 275 no livro-texto
          \cite{stallings} aponta que são casos extremos onde o processador irá
          demorar mais que o geralmente para fazer uma exponenciação modular,
          esse mesmo é largamente usado pois é mais eficiente para o computador
          quando se fala de números grandes. Para alguns valores de $a$ e $d$ na
          operação $d \gets (d \times a)$ a multiplicação vai ser lenta, e o
          atacante sabe quais são esses valores os quais serão lentos [KOCH96].
            \begin{enumerate}
              \item Esse exemplo pode ser encontrado na página 276 da literatura
              \cite{stallings}.
              \item Gere um valor aleatório r entre 0 e $n - 1$.
              \item Calcule $C^{'} = C(r^e) \ (\textrm{mod}\ n)$ \textbf{e} é o 
              expoente público.
              \item Calcule $M^{'} = {(C^{'})}^d \ (\textrm{mod}\ n)$ com o 
              próprio algoritmo RSA.
              \item Calcule $M = M^{'} \times d \ (\textrm{mod}\ n)$ \textbf{d}
              é a inversa multiplicativa de $r \ (\textrm{mod}\ n)$.
              \item RSA reporta que de 2 a 10\% da performance desse algoritmo
              sofre penalidade por causa de \textbf{blinding} (ciphertext x 
              random number, para aumentar a confusão).
            \end{enumerate}
          \item \textbf{Hardware fault-based attack}: Esse envolve em induzir
          possibilidade de falhas no processador que está gerando assinaturas
          digitais.
          \item \textbf{Chosen ciphertext attacks}: Esse tipo de ataque
          explora propriedade do algoritmo RSA. Página 277 \cite{stallings}
            \begin{enumerate}
              \item Um simples exemplo de CCA contra RSA, o algoritmo pode ser
              encontrando na página 277 \cite{stallings}:
              \item Podemos tentar decifrar $C = M^e \ (\textrm{mod}\ n)$
              \item Isto é, computar $X = (C \times 2^e) \ (\textrm{mod}\ n)$
              \item Submeter $X$ como um texto-cifrado e receber $Y = X^d \
              (\textrm{mod}\ n)$
              \item Teremos: $X  = (C \ \textrm{mod}\ n) \times (2^e \
              \textrm{mod}\ n) = (M^e \ \textrm{mod}\ n) \times (2^e \
              \textrm{mod}\ n) = ((2 \times M)^e \ \textrm{mod}\ n)$
              \item Chegamos à $Y = (2M) \ \textrm{mod}\ n$
              \item M é dedutível.
            \end{enumerate}
        \end{enumerate}
    \end{enumerate}

  \item Por que é importante ter-se bons gerados de números aleatórios para
  gerar chaves RSA ?

    \begin{enumerate}
      \item Uma análise comparando milhões de chaves publicas em 2012 por Arjen
      K. Lenstra, James P. Hughes, Maxime Augier, Joppe W. Bos, Thorsten
      Kleinjung and Christophe Wachter (Criptanalistas). Eles foram capazes de
      fatorar apenas 0.2\% dessas mesmas usando o algoritmo de Euclides.
      \cite{markoff} \cite{ronwaswrong}
      
      \item Ele usando uma artimanha, veja, se $n = pq$ é uma chave pública e
      $n^{'} = p^{'}q^{'}$ é outras, então se por alguma chance $p = p^{'}$ (mas
      $q$ não igual à $q^{'}$), então uma simples computação do $gcd(n, n^{'}) =
      p$ fatora ambos $n$ e $n^{'}$, comprometendo ambas as chaves.
      
      \item Observe que você pode minimizar esse problema, basta utilizar um
      gerador de \textit{sementes} com forte tendencia aleatória duplica o nível
      de segurança. Ou adotar uma função determinística para escolha de $q$ dado
      $p$, ao invés de escolher $p$ e $q$ independentemente.
      
      \item Gerador forte de números aleatórios é importante através de qualquer
      fase na geração de chave pública. Veja que, se um gerador fraco de números
      aleatórios for utilizado para chaves simétricas que estão sendo
      distribuídas pelo algoritmo de RSA, então um atacante poderia transpassar
      o algoritmo de RSA e adivinhar diretamente as chaves simétricas.
    \end{enumerate}

\end{enumerate}

\begin{thebibliography}{1}

\bibitem{markoff} Markoff, John (February 14, 2012). "Flaw Found in an Online
Encryption Method". New York Times. Disponível em: http://www.nytimes.com/20\\
12/02/15/technology/researchers-find-flaw-in-an-online-encryption-method.html

\hfil

\bibitem{ronwaswrong} "Ron was wrong, Whit is right". Disponível em:
\href{http://eprint.iacr.org/2012/064.pdf}{pdf}.

\hfil

\bibitem{carm} "Carmichael Function". Disponível no:
\href{http://mathworld.wolfram.com/CarmichaelFunction.html}{link}

\hfil

\bibitem{pkcs1} "The PKCS \#1 standard". Disponível no: 
\href{https://en.wikipedia.org/wiki/PKCS\_1#Keys}{link}

\hfil

\bibitem{rsakey} "RSA Cryptosystem Key Generation". Disponível no:
\href{https://en.wikipedia.org/wiki/RSA\_cryptosystem\#Key\_generation}{link}

\hfil

\bibitem{timat} Remote timing attacks are practical. SSYM'03 Proceedings of the
12th conference on USENIX Security Symposium. Disponível em:
\href{http://crypto.stanford.edu/~dabo/papers/ssl-timing.pdf}{pdf}

\hfil

\bibitem{stallings} Stallings, W. Cryptography and Network Security:
Principles and Practice, 6th ed. Upper Saddle River, NJ: Prentice Hall, 2013,
752p., \$142.40. ISBN: 13: 978-0133354690.

\end{thebibliography}

\smallskip

\end{document}
