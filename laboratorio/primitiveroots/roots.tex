\documentclass[conference]{IEEEtran}
\usepackage{cite, amsmath, graphicx, hyperref, algorithm}
\usepackage[utf8]{inputenc}
\usepackage[brazil, english]{babel}
\usepackage[noend]{algpseudocode}

\title{Algorítmo Gerador das Raízes Primitivas Menores que um Número Inteiro}

\author {
  Lucas Tonussi\\
  \textsc{lptonussi@gmail.com}\\
  \small{Ciência da Computação}
}

\begin{document}

\maketitle

\begin{abstract}
Criação de um gerador das raízes primitivas menor que um inteiro qualquer.
\end{abstract}

\IEEEoverridecommandlockouts

\begin{keywords}
Raízes Primitivas, Maior Divisor Comum, Aritmética Modular
\end{keywords}

\IEEEpeerreviewmaketitle

\section{Introdução}

Esse artigo mostra os algorítimos em forma de pseudo-código, das funções usadas
no programa \textit{PrimitiveRoots} que acompanha esse documento. A linguagem é
em python pela facilidade em traduzir pseudo-código teórico para um objeto
executável. Basicamente para rodar o código basta no seu terminal (unix/windows)
com Python 2 basta executar \textsc{python proots.py}. Com Python 3 no seu
terminal basta escrever \textsc{python3 roots.py}.

\section{Calculo do Totient $\phi(n)$}

O cálculo do TOTIENT $\phi(n)$ é feito usando o algoritmo de Euclides
\cite{totient}, que consta na seção \textit{Computing Euler's function} na
Wikipédia. Assim como também o $MDC_{RECURSIVO}$ é o algoritmo de Euclides
\cite{gcd}.

\begin{algorithm}
  \caption{Função de Euler para $MDC_{RECURSIVO}$}\label{gcd}
  \begin{algorithmic}
    \Function{$MDC_{RECURSIVO}$}{N : Integer, k : Integer}
      \If {$k == 0$}
        \State \Return N
      \EndIf
      \If {$N == 0$}
        \State \Return k
      \EndIf
      \Return $MDC_{RECURSIVO}(k, N \div k)$
    \EndFunction
  \end{algorithmic}
\end{algorithm}

Abaixo está o algoritmo de Euclides para calcular $\phi(n)$. Essa função é para
utilizar como comparação da quantidades de raízes primitivas achadas de dado um
número de entrada N, primo no Algoritmo \ref{proots}. Ou seja, fazer
totient(totient(n)) tem que dar igual ao comprimento do arranjo que guarda todos
as raízes primitivas computadas no Algoritmo \ref{proots}.

\begin{algorithm}
  \caption{Função de Euler de Totient}\label{totient}
  \begin{algorithmic}
    \Function{totient}{N : Integer}
      \State {$m \gets 0$}
      \State {$k \gets 1$}
      \While {$k < n + 1$}
        \If {$MDC_{RECURSIVO}(N, k) == 1$}
          \State $m \gets m + 1$
        \EndIf
        \State {$k \gets k + 1$}
      \EndWhile
      \Return m
    \EndFunction
  \end{algorithmic}
\end{algorithm}

\section{Teste de Primo}

Para calcular se dado número inteiro é primo \cite{primtest} usei um algoritmo
em pseudo código que usa aritmética modular para chegar na resposta.

\begin{algorithm}
  \caption{Testador de Número Primo}\label{isprime}
  \begin{algorithmic}
    \Function{testaPrimo}{N : Integer}
    \If {$N \le 1$}
      \Return False
    \EndIf
    \If {$N \le 3$}
      \Return True
    \EndIf
    \If {$N \div 2 = 0 \textbf{ or } N \div 3 = 0$}
      \Return False
    \EndIf
    \State $i \gets 5$
    \While {$i \times i \le n$}
      \If {$N \div i = 0 \textbf{ or } N \div (i + 2) = 0$}
        \State \Return False
      \EndIf
      \State $i \gets i + 6$
    \EndWhile
    \Return True
    \EndFunction
  \end{algorithmic}
\end{algorithm}

\section{Computando Raízes Primitivas}

O Algoritmo de Divisão Trivial serve para achar os fatores primos de um número.
Juntamento com seu conjunto de primos de 2 até $n - 1$.

\begin{algorithm}
  \caption{Divisão por Tentativas} \label{trialdiv}
  \begin{algorithmic}
    \Function{divisaoTentativas}{primos : Lista, n : Inteiro}
      \If {$n < 2$}
        \Return ConjuntoVazio
      \EndIf
      \State {fatores = Conjunto}
      \For {i de 0 até $comprimento(primos)$}
        \While {$n \div p = 0$}
          \State {$fatores \gets primos[i]$}
          \State {n = n / p}
        \EndWhile
      \EndFor
      \If {$n > 1$}
        \State {$fatores \gets n$}
      \EndIf
      \Return fatores
    \EndFunction
  \end{algorithmic}
\end{algorithm}

Segundo consta na Wikipédia \cite{primtest} uma computação eficaz para achar as
raízes primitivas de um número inteiro positivo maior é bastante complicada
\cite{robbins}\cite{vonzur}. Observe que $testaPrimo(n)$ está explicado na seção
anterior. Considerar que a atribuição de $Conjunto$ abaixo mantém valores únicos
dentro do arranjo (não repetidos). O valor de N deve ser primo caso contrário o
Algorítimo falha de início. Antes de testar se os valores de m (Naturais) de 2
até N o Algorítimo fatora em primos o valor de $phi(N) = N - 1$. Eu inverti a
lógica de teste diferente à 1 pois eu tenho que testar se todo conjunto de
fatores computados dão igual à diferente de 1, e aí sim proclamar o valor $m$
sendo testado, como um valor raiz primitiva e então armazenar no conjunto de
raízes primitivas $primitiveRoots$.

\begin{algorithm}
  \caption{Computando Raízes Primitivas}\label{proots}
  \begin{algorithmic}
    \Function{raizesPrimitivas}{N : Integer}
      \If {N não é primo}
        \State \Return {"Entrada precisa ser um nro primo"}
      \EndIf
      \State {$totiente \gets N - 1$}
      \State $fatores \gets Conjunto$
      \State $primos \gets Conjunto$
      \State $raizesPrimitivas\gets Conjunto$
      \For {v from 2 to n - 1}
        \If {$testaPrimo(n)$}
          \State{$primos \gets v$}
        \EndIf
      \EndFor
      \State {$fatores \gets divisaoTentativas(primos,
      totient)^{\ref{trialdiv}}$}
      \For {m de 2 até n}
        \State {$todosRaiz \gets True$}
        \For {para todo $pf$ no conjunto $fatores$}
          \If {$m^{\frac{\phi(n)}{pf}} \div n = 1$}
            \State {$todosRaiz \gets False$}
            \State {\textbf{break}}
          \EndIf
        \EndFor
        \If {todosRaiz = True}
          \State $raizesPrimitivas \gets m$
        \EndIf
      \EndFor
      \Return raizesPrimitivas
    \EndFunction
  \end{algorithmic}
\end{algorithm}

O algoritmo para computar as raízes de um valor inicial N (inteiro), usa um
\textit{hack} de um Criptografo chamado Bruce Schneier. O pseudo-código a seguir
é baseado no livro \textit{Applied Cryptography} de Bruce Schneier
\cite{schneier}. E dado uma base, expoente, e um valor para módulo
respectivamente \textbf{b}, \textbf{e}, e \textbf{m} o algorítimo computa $b^e
(\div m)$, exemplo se b = 4, e = 13, m = 497, note que o expoente é 1011 em
binário, sendo ele 4 digito o algoritmo não deve demorar mais que 4 iterações
para responder o valor 445 \cite{powermod}. \\

\begin{algorithm}
  \caption{Base, Potência e Modulo}\label{powmod}
  \begin{algorithmic}
    \Function{moduloPotencia}{base : Integer, expoente : Integer, modulo :
    Integer}
      \If {$(m - 1) * (m - 1) \text{ does not overflow base}$}
        \State \textbf{raise} \textit{OverflowInteger64Exception}
      \EndIf
      \State {$result \gets 1$}
      \State {$base \gets base \div modulo$}
      \While {$exponent < 0$}
        \If {$expoente \div 2 = 1$}
          \State {$resultado \gets (resultado \times base) \div modulo$}
        \EndIf
        \State {$expoente \gets expoente \gg 1$}
        \State {$base \gets (base \times base) \div modulo$}
      \EndWhile
      \Return resultado
    \EndFunction
  \end{algorithmic}
\end{algorithm}

\section{Conclusões}

Problema do algoritmo de fatoração é que ele é muito ruim para grandes números
quebrando toda sequencia do algoritmo de testar raízes primitivas. Existem
outros métodos de fatoras em primos porém eu não consegui implementar, e optei
pelo método comumente conhecido de fatoração por tentativas. O Algoritmo do
Bruce Schneier \cite{schneier} é bom para números grandes visto que ele trabalha
melhor o expoente da base, usando deslocamento para esquerda.

\begin{thebibliography}{1}
\bibitem{powermod} Pseudocode basiado no livro Applied Cryptography de Bruce
Schneier.; Wikipédia disponível em:
\href{http://en.wikipedia.org/wiki/Modular_exponentiation#Right-to-left_binary_method}{http://en.wikipedia.org/wiki/Modular\_exponentiation\#Right-to-left\_binary\_method}

\bibitem{gcd} Euler Greatest Common Divisor Algorithm.; Wikipédia disponível em:
\href{https://en.wikipedia.org/wiki/Euclidean_algorithm#Euclidean_division}{https://en.wikipedia.org/wiki/Euclidean\_algorithm\#Euclidean\_division}

\bibitem{totient} Euler Totient Function.; Wikipédia disponível em:
\href{https://en.wikipedia.org/wiki/Euler\%27s_totient_function#Computing_Euler.27s_function}{https://en.wikipedia.org/wiki/Euler's\_totient\_function}

\bibitem{primtest} Pseudo-código para teste de números primos.; Wikipédia
disponível em:
\href{https://en.wikipedia.org/wiki/Primality_test#Pseudocode}{https://en.wikipedia.org/wiki/Primality\_test\#Pseudocode}

\bibitem{citekey} Raíz primitiva módulo N.; Wikipédia disponível em:
\href{http://en.wikipedia.org/wiki/Primitive_root_modulo_n}{http://en.wikipedia.org/wiki/Primitive\_root\_modulo\_n}

\bibitem{vonzur} von zur Gathen, Joachim; Shparlinski, Igor (1998), "Orders of
Gauss periods in finite fields", Applicable Algebra in Engineering,
Communication and Computing 9 (1): 15–24, doi:10.1007/s002000050093, MR 1624824,
"One of the most important unsolved problems in the theory of finite fields is
designing a fast algorithm to construct primitive roots".

\bibitem{robbins} Robbins, Neville (2006), Beginning Number Theory, Jones \&
Bartlett Learning, p. 159, ISBN 9780763737689, "There is no convenient formula
for computing [the least primitive root]".

\bibitem{schneier} Schneier 1996, p. 244.
\end{thebibliography}

\smallskip

\end{document}
